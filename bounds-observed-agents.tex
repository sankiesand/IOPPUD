%!TEX root = main.tex
\section{Bounds on the number of observed agents}
\label{sec:bounds-observed-agents}

\begin{definition}
	Let $\run : \config_1 \xrightarrow{*} \config_2$ be a "run".
	We say that agent $a_0$ is ""internally observed"" in $\run$ if $\run$ contains a step of the form $\config_1 \step{=}{a}{a_o} \config_2$, and ""externally observed"" if $\run$ contains a step of the form $\config_1 \step{\neq}{a}{a_o} \config_2$.
	We say that $a_o$ is ""observed"" in $\run$ if it is either "internally" or "externally observed", and that a datum $d$ is observed if an agent $a$ with $\dataof(a) =d$ is. 
\end{definition}

\subsection{Bounds on the number of observed agents per datum}

We call  ""trajectory"" of an agent $a$ in a run $\run=C_1 \xrightarrow{} C_2 \xrightarrow{} \ldots \xrightarrow{} C_n$ 
the sequence $q_1 \ldots q_n$ of states of $Q$ such that  $C_i(a)=q_i$ for all $i$.

\begin{lemma}[Agents core lemma]
	\label{lem:agents-core-lemma}
	Let $\run : \config_1 \xrightarrow{*} \config_2$ be a "run" with $\Agentset_{\run}$ the set of agents appearing in it. There exists a "run" $\run' : \config'_1 \xrightarrow{*} \config'_2$ over a subset of agents $\Agentset_{\run'} \subseteq \Agentset_{\run}$ such that:
	\begin{itemize}
		\item for all $a \in \Agentset_{\run'}$, $\config_1(a) = \config'_1(a)$ and $\config_2(a) = \config'_2(a)$,
		
		\item and for all $d \in \Dataset$, $q_1, q_2 \in Q$, if there exists an agent in $\Agentset_{\run}$ such that $\config_1(a) = (q_1, d)$ and $\config_2(a) = (q_2, d)$ then there is such an agent in $\Agentset_{\run'}$.
		
		\item For all $d \in \Dataset$, there are at most $\size{Q}^3$ agents with datum $d$ in $\Agentset_{\run'}$.
		%\cortoin{we can achieve $\size{Q}^3$ but I think it makes the proof a little more technical}
	\end{itemize}
\end{lemma}

\begin{proof}
%	We mimic the \emph{bunch} argument from \cite{EsparzaRW2019}. 
%	Let $d\in \Dataset$ a datum appearing in $\run$, and let $q_1, q_2 \in Q$, and $\Agentset^d_{q_1, q_2} = \set{a \in \Agentset_{\run} \mid \dataof(a) = d, \config_1(a) = q_1, \config_2(a) = q_2}$. Suppose $\size{\Agentset^d_{q_1, q_2}} \geq 3\size{Q}$, then let $S$ be the set of states visited by agents of $\Agentset^d_{q_1, q_2}$ in $\run$.
%	
%	For all $q \in S$ let $\alpha_q$ be the first agent of $\Agentset^d_{q_1, q_2}$ to reach $q$ (for $q_1$ we pick one arbitrarily), and $\beta_q$ the last one to leave it (we pick an arbitrary agent for $\beta_{q_2}$). 
%	Note that those agents do not have to be distinct.
%	
%	We pick a family of distinct agents $(a_q)_{q \in S}$ in $\Agentset^d_{q_1, q_2}$ that is disjoint from $(\alpha_q)_{q \in S}$ and $(\beta_q)_{q \in S}$ (we can do so as $\size{\Agentset^d_{q_1, q_2}} \geq 3\size{Q}$).
%	We define a new "run" $\overline{\run}$, where all agents not in $\Agentset^d_{q_1, q_2}$, as well as all $\alpha_q$ and $\beta_q$ behave the same. The agents of $\Agentset^d_{q_1, q_2}$ besides the $\alpha_q, \beta_q$ and $a_q$ are deleted.
%	Each $a_q$ mimics the transitions taken by $\alpha_q$ until it reaches $q$, then stays idle until $\beta_q$ reaches $q$ for the last time, after which it mimics $\beta_q$. \cortoin{informal, formal proof by induction on the "run"}
%	
%	Whenever an agent "observes" an agent of $\Agentset_{q_1, q_2}$ in state $q$, as this must happen in $\run$ between the arrival of $\alpha_q$ and the departure of $\beta_q$, $a_q$ is in $q$ at this point in $\overline{\run}$ and can hence be "observed" instead. Therefore all "observations" are still possible, and only agents of $(a_{q})_{q\in S}$ are "observed" among $\Agentset_{q_1, q_2}$.
%	
%	The start and end "configurations" are the same as before for all remaining agents as all $a_q$ still go from $q_1$ to $q_2$, and other agents behave the same.
%	
%	By applying this transformation on all data appearing in $\run$ and all pairs of state $q_1, q_2$, we obtain a "run" $\run'$ in which for all $d$, $q_1$, $q_2$, at most $3\size{Q}$ agents going from $q_1$ to $q_2$ with datum $d$ remain (either there are less than $3\size{Q}$ such agents, or we ensured than at most $\size{Q}$ of them are "observed").
%	In total, for all $d$, at most $3\size{Q}^3$ agents per datum remain.
%%%%%%%%%%%%%%%%%%%%%%%%%%%%%%   old proof above
We mimic the \emph{bunch} argument from \cite{EsparzaRW2019}. \chana{old proof commented out}
Let $\run$ be the run $\config_1=C_1 \xrightarrow{} C_2 \xrightarrow{} \ldots \xrightarrow{} C_n = \config_2$.
Let $d\in \Dataset$ a datum appearing in $\run$,  let $q_1, q_2 \in Q$, 
Suppose $\size{\Agentset^d_{q_1, q_2}} \geq \size{Q}$, and 
let $\bunchreach$ be the set of states visited by agents of $\Agentset^d_{q_1, q_2}$.
We are going to define a family $(a_q)_{q \in \bunchreach}$ of agents in $\Agentset^d_{q_1, q_2}$ 
such that the run with only  $(a_q)_{q \in \bunchreach}$ in $\Agentset^d_{q_1, q_2}$ 
and equal to $\run$ everywhere else is still a valid run,
Repeating the operation for every set of agents $\Agentset^d_{q_1, q_2}$
will yield a run fulfilling the Lemma conditions.

For all $q \in \bunchreach$, 
let $f$ be the first moment $q$ is reached in $\run$, 
i.e. the minimal index such that there exists an $a\in \Agentset^d_{q_1, q_2}$
with $C_f(a)=q$.
Let $l$ be the last moment $q$ is reached in $\run$, 
i.e. the maximal index such that there exists an $a\in \Agentset^d_{q_1, q_2}$
with $C_l(a)=q$.
Let $\alpha_q$ be one of the agent in $\Agentset^d_{q_1, q_2}$ that reaches $q$ first,
i.e. $C_f(\alpha_q)=q$, and 
let $\beta_q$ be one of the agent in $\Agentset^d_{q_1, q_2}$ that reaches $q$ last
i.e. $C_l(\beta_q)=q$. 
Note that these agents do not have to be distinct.
We pick an agent $a_q \in \Agentset^d_{q_1, q_2}$ and modify its trajectory as follows.
Agent $a_q$ mimics the transitions taken by $\alpha_q$ until $f$, 
then stays idle until $l$, after which it mimics $\beta_q$. 
We can choose the $a_q$ distinct for each $q\in \bunchreach$, since $\size{\Agentset^d_{q_1, q_2}} \geq \size{Q}$.

We modify $\run$ by deleting all agents of $\Agentset^d_{q_1, q_2}$ that are not in $(a_q)_{q \in \bunchreach}$.
This is still a valid run, because whenever an agent "observes" an agent of $\Agentset^d_{q_1, q_2}$ in state $q$, 
 this must happen at moment $i$ between $f$ and $l$. 
 Our definition of $a_q$'s trajectory ensures it  is in $q$ at $i$ and can hence be "observed" instead. 
 Therefore all "observations" are still possible.
	
By applying this transformation on all data appearing in $\run$ and all pairs of state $q_1, q_2$, 
we obtain a "run" $\run'$ in which for all $d$, $q_1$, $q_2$, 
at most $\size{Q}$ agents going from $q_1$ to $q_2$ with datum $d$ remain 
In total, for all $d$, at most $\size{Q}^3$ agents per datum remain.
\chana{tbd: show $\run'$ fulfills the lemma conditions}
\chanain{other option: for all $d$, treat $\run_d$ (only the $d$-agents of $\run$) as a classic IO run, and apply the results of \cite{EsparzaRW2019}}
\end{proof}



\begin{lemma}[Agents copycat lemma]
	\label{lem:agents-copycat}
	Let  $\run : \config_1 \xrightarrow{*} \config_2$ a "run", let $q_1, q_2 \in Q$,  $d \in \Dataset$, and  $\Tilde{a} \in \Agentset$ such that $\dataof(\Tilde{a}) = d$ and $\config_1(\Tilde{a}) = \bot$. 
	
	If there exists an agent $a$ such that $\dataof(a)=d$, $\config_1(a) = q_1$ and $\config_2(a) = q_2$, then there exists a run $\Tilde{\run} : \Tilde{\config}_1 \xrightarrow{*} \Tilde{\config}_2$ such that
	\begin{itemize}
		\item $\Tilde{\config}_1(\Tilde{a}) = q_1$, and $\Tilde{\config}_1(a') = \config_1(a')$ for all $a' \neq \Tilde{a}$.
		
		\item $\Tilde{\config}_2(\Tilde{a}) = q_2$, and $\Tilde{\config}_2(a') = \config_2(a')$ for all $a' \neq \Tilde{a}$.
		
		\item $\Tilde{a}$ is not observed in $\Tilde{\run}$.
	\end{itemize}
\end{lemma}

\begin{proof}
	We set $\Tilde{\config}_1$ such that $\Tilde{\config}_1(\Tilde{a}) = q_1$ and $\Tilde{\config}_1(a') = \config_1(a')$ for all $a' \neq \Tilde{a}$.
	
	We then construct $\Tilde{\run}$ by inserting after every step $\step{\sim}{a}{a_o}$ in $\run$ during which $\Tilde{a}$ moves a step
	$\step{\sim}{\Tilde{a}}{a_o}$.
	We can do so because $\dataof(\Tilde{a})=\dataof(a)$, and by immediate induction they are in the same state before every step $\step{\sim}{\Tilde{a}}{a_o}$.
	
	The configuration $\Tilde{\config}_2$ reached by $\Tilde{\run}$ is such that $\Tilde{\config}_2(\Tilde{a}) = \Tilde{\config}_2(a) = q_2$ and $\Tilde{\config}_1(a') = \config_1(a')$ for all $a' \neq \Tilde{a}$. The two first items of the lemma are trivially satisfied.
	For the third one, observe that $\Tilde{a}$ did not appear in $\run$ and that we did not add any transition where $\Tilde{a}$ is observed.
\end{proof}


\begin{corollary}
	For all "run" $\run : \config_1 \xrightarrow{*} \config_2$, there exists a "run" $\Tilde{\run} : \config_1 \xrightarrow{*} \config_2$ such that for all $d \in \Dataset$,at most $3\size{Q}^3$ agents with datum $d$ are "observed" in $\run$.
\end{corollary}

\begin{proof}
	Take the "run" $\run'$ from~\cref{lem:agents-core-lemma} and add the missing agents using \cref{lem:agents-copycat} by making them copycat an agent with the same datum, initial and final state from $\Agentset_{\run'}$. Such an agent exists by the second item of the lemma.
\end{proof}

\subsection{Bounds on the number of observed data}

\begin{lemma}[Data core lemma]
	\label{lem:data-core-lemma}
	Let $\run : \config_0 \to \config_2 \to \cdots \to \config_m$ be a "run"  over a set of data $\Dataset_{\run}$, and let $K$ be such that there are at most $K$ "observed" agents of each datum "observed" in the "run". There exists another "run" $\run' : \config_0 \xrightarrow{*} \config_m$ and a subset of data $\Dataset_{\run'}$ such that:
	\begin{itemize}
		\item for all agents $a$ of datum $d \in \Dataset_{\run'}$, $\config_1(a) = \config'_1(a)$ and $\config_2(a) = \config'_2(a)$,
		
		\item for all $d \in \Dataset_{\run}$, there exists $d' \in \Dataset_{\run'}$ such that $\shadow{d'}{\run'} = \shadow{d}{\run}$.
		
		\item $\size{\Dataset_{\run'}} \leq 4(K+2)^{2\size{Q}^2}$.
	\end{itemize}
\end{lemma}


\begin{proof}
	We proceed similarly to the proof of \cref{lem:agents-core-lemma}. 
	Let $\Dataset_\run$ be the set of data appearing in $\run$.
	For all $d$ let $\Agentset_d$ be its set of agents and $\Agentset_{d,o}$ the subset of those agents that are "observed" at some point in the "run".
	
	As no more than $K$ agents of each datum are "observed" through $\run$, the "shadows" of all data are bounded by $K$ and thus there are at most $(K+2)^{\size{Q}^2}$ different ones. This also holds for all prefixes and suffixes of $\run$.
	
	Let $M=(K+2)^{2\size{Q}^2}$. 
	For all $d \in \Dataset_\run$ and $i \in \nset{1}{m}$ we define $\sigma(d,i) = (\shadow{d}{\prefixrun{\run}{i}}, \shadow{d}{\suffixrun{\run}{i}})$. This functions takes a datum and a point in the run and returns the "shadow" of the past and the future of that datum through the run. 
	
	We now lift the proof of \cref{lem:agents-core-lemma} from agents to data.
	Let $\ashadow : Q^2 \to \nset{0}{K}\cup \bot$ and let $\Dataset_{\ashadow} = \set{d \in \Dataset_{\run} \mid \shadow{d}{\run} = \ashadow}$. 
	Suppose $\size{\Dataset_{\ashadow}} \geq 4M$. 
	Let $S = \set{\sigma(d,i) \mid d \in \Dataset_{\ashadow}, i\in\nset{1}{m}}$. Note that $\size{S}\leq M$.
	
	For each $s \in S$ define $\alpha_s$ as the datum reaching $s$ first (i.e., such that $\sigma(\alpha_s,i) = s$ for a $i$ as small as possible) in $\run$ and $\beta_s$ as the last one.
	Then as $\size{\Dataset_{\ashadow}}\geq 4M$, we can pick for each $s \in S$ some data $d_s, e_s \in \Dataset_{\ashadow}$ such that the $(d_s)_{s\in S}, (e_s)_{s\in S}$ are disjoint and disjoint from the $(\alpha_s)_{s\in S}$ and $(\beta_s)_{s\in S}$.
	
	We create a new "run" $\overline{\run}$ where all agents with datum in $\set{\alpha_s, \beta_s \mid s \in S} \cup (\Dataset \setminus \Dataset_{\ashadow})$ behave the same. 
	The agents with datum in $\Dataset_{\ashadow} \setminus \set{\alpha_s, \beta_s, d_s, e_s \mid s \in S}$ are deleted.
	
	Let $s = (\ashadow_1, \ashadow_2) \in S$, we make agents of $d_s$ and $e_s$ follow the ones of $\alpha_s$ until it reaches $s$, then stay idle until $\beta_s$ last reaches $s$, and then follow $\beta_s$.
	This is possible as they have the same "shadows", hence we can match "observed" agents of $d_s$ (and $e_s$) to the ones of $\alpha_s$ (resp. $\beta_s$) one-to-one while preserving the states of departure and arrival, and the "non-observed" of $d_s$ and $e_s$ can follow agents of $\alpha_s$ (resp. $\beta_s$) that have the same starting and finishing states.
	
	When an agent with datum $d$ is "externally observed" at step $i$ in $\run$, we consider $s = \sigma(d,i)$ and make the moving agent "observe" $d_s$ or $e_s$ instead (one of them has to be different from $d$).
	
	We apply this transformation on all "shadows".
\end{proof}




\begin{lemma}[Data copycat lemma]
	\label{lem:data-copycat}
	Let  $\run : \config_1 \xrightarrow{*} \config_2$ a "run", let $\ashadow : Q^2 \to \nats \cup \set{\bot}$ a "shadow", $\atrace$ a trace such that $\atrace(q_1, q_2) \geq \ashadow(q_1, q_2)$ for all $q_1, q_2$ such that $\ashadow(q_1, q_2) \in \nats$ and let $\Tilde{d} \in \Dataset$ such that $\config_1$ does not contain any agent with datum $\Tilde{d}$. 
	
	If there exists a datum $d$ such that $\shadow{\run}{d}=\ashadow$ then there exists a run $\Tilde{\run} : \Tilde{\config}_1 \xrightarrow{*} \Tilde{\config}_2$ such that
	\begin{itemize}
		\item $\shadow{\Tilde{\run}}{\Tilde{d}} = \ashadow$ and $\shadow{\Tilde{\run}}{d'} = \shadow{\Tilde{\run}}{d'}$ for all $d' \neq \Tilde{d}$.
		
		\item $\trace{\run}{\Tilde{d}} = \atrace$ and $\trace{\run}{d'} = \trace{\Tilde{\run}}{d'}$ for all $d' \neq \Tilde{d}$.
	\end{itemize}
\end{lemma}

\begin{corollary}
	\label{cor:run-few-observed-data-and-agents}
	For all "run" $\run : \config_1 \xrightarrow{*} \config_2$, there exists a "run" $\Tilde{\run} : \config_1 \xrightarrow{*} \config_2$ such that for all $d \in \Dataset$,at most $3\size{Q}^3$ agents with datum $d$ are "observed" in $\run$, and agents of at most $4(3\size{Q}^3+2)^{2\size{Q}^2}$ different data are "observed".
\end{corollary}



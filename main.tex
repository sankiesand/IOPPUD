\documentclass[a4paper,UKenglish,cleveref, autoref, thm-restate]{lipics-v2021}
%This is a template for producing LIPIcs articles. 
%See lipics-v2021-authors-guidelines.pdf for further information.
%for A4 paper format use option "a4paper", for US-letter use option "letterpaper"
%for british hyphenation rules use option "UKenglish", for american hyphenation rules use option "USenglish"
%for section-numbered lemmas etc., use "numberwithinsect"
%for enabling cleveref support, use "cleveref"
%for enabling autoref support, use "autoref"
%for anonymousing the authors (e.g. for double-blind review), add "anonymous"
%for enabling thm-restate support, use "thm-restate"
%for enabling a two-column layout for the author/affilation part (only applicable for > 6 authors), use "authorcolumns"
%for producing a PDF according the PDF/A standard, add "pdfa"

%\pdfoutput=1 %uncomment to ensure pdflatex processing (mandatatory e.g. to submit to arXiv)
%\hideLIPIcs  %uncomment to remove references to LIPIcs series (logo, DOI, ...), e.g. when preparing a pre-final version to be uploaded to arXiv or another public repository

%\graphicspath{{./graphics/}}%helpful if your graphic files are in another directory

\bibliographystyle{plainurl}% the mandatory bibstyle

\title{Immediate Observation Population Protocols with Unordered Data}

\titlerunning{IOPPUD} %TODO optional, please use if title is longer than one line

%\author{Jane {Open Access}}{Dummy University Computing Laboratory, [optional: Address], Country \and My second affiliation, Country \and \url{http://www.myhomepage.edu} }{johnqpublic@dummyuni.org}{https://orcid.org/0000-0002-1825-0097}{(Optional) author-specific funding acknowledgements}%TODO mandatory, please use full name; only 1 author per \author macro; first two parameters are mandatory, other parameters can be empty. Please provide at least the name of the affiliation and the country. The full address is optional. Use additional curly braces to indicate the correct name splitting when the last name consists of multiple name parts.

\author{Joan R. Public}{Department of Informatics, Dummy College, Country}{joanrpublic@dummycollege.org}{}{}


\authorrunning{} %TODO mandatory. First: Use abbreviated first/middle names. Second (only in severe cases): Use first author plus 'et al.'

\Copyright{Joan R. Public} %TODO mandatory, please use full first names. LIPIcs license is "CC-BY";  http://creativecommons.org/licenses/by/3.0/

%\ccsdesc[100]{\textcolor{red}{Replace ccsdesc macro with valid one}}
%%TODO mandatory: Please choose ACM 2012 classifications from
%%https://dl.acm.org/ccs/ccs_flat.cfm

% \begin{CCSXML}
	% 	<ccs2012>
	% 	<concept>
	% 	<concept_id>10003752.10003753.10003761.10003763</concept_id>
	% 	<concept_desc>Theory of computation~Distributed computing models</concept_desc>
	% 	<concept_significance>500</concept_significance>
	% 	</concept>
	% 	</ccs2012>
	% \end{CCSXML}

\ccsdesc[500]{Theory of computation~Verification by model checking}
\ccsdesc[300]{Theory of computation~Distributed computing models}

% \ccsdesc{Theory of computation~Models of computation~Concurrency~Distributed computing models}

\keywords{Population protocols} %TODO mandatory; please add comma-separated list of keywords



%\category{} %optional, e.g. invited paper

%\relatedversion{} %optional, e.g. full version hosted on arXiv, HAL, or other respository/website
%\relatedversiondetails[linktext={opt. text shown instead of the URL}, cite=DBLP:books/mk/GrayR93]{Classification (e.g. Full Version, Extended Version, Previous Version}{URL to related version} %linktext and cite are optional

%\supplement{}%optional, e.g. related research data, source code, ... hosted on a repository like zenodo, figshare, GitHub, ...
%\supplementdetails[linktext={opt. text shown instead of the URL}, cite=DBLP:books/mk/GrayR93, subcategory={Description, Subcategory}, swhid={Software Heritage Identifier}]{General Classification (e.g. Software, Dataset, Model, ...)}{URL to related version} %linktext, cite, and subcategory are optional

%\funding{(Optional) general funding statement \dots}%optional, to capture a funding statement, which applies to all authors. Please enter author specific funding statements as fifth argument of the \author macro.

%\acknowledgements{I want to thank \dots}%optional

%\nolinenumbers %uncomment to disable line numbering



%Editor-only macros:: begin (do not touch as author)%%%%%%%%%%%%%%%%%%%%%%%%%%%%%%%%%%
% \EventEditors{John Q. Open and Joan R. Access}
% \EventNoEds{2}
% \EventLongTitle{42nd Conference on Very Important Topics (CVIT 2016)}
% \EventShortTitle{CVIT 2016}
% \EventAcronym{CVIT}
% \EventYear{2016}
% \EventDate{December 24--27, 2016}
% \EventLocation{Little Whinging, United Kingdom}
% \EventLogo{}
% \SeriesVolume{42}
\ArticleNo{1}
%%%%%%%%%%%%%%%%%%%%%%%%%%%%%%%%%%%%%%%%%%%%%%%%%%%%%%




\newcommand{\cortoin}[1]{\todo[color=blue!20,inline]{\small #1}}
\newcommand{\corto}[1]{\todo[color=blue!20]{\small #1}}

\newcommand{\nicoin}[1]{\todo[color=red!20,inline]{\small #1}}
\newcommand{\nico}[1]{\todo[color=red!20]{\small #1}}


\newif\ifproofs
\proofstrue

\newif\ifintuition
\intuitionfalse



\usepackage[utf8]{inputenc}
\usepackage{xcolor}
\usepackage{hyperref}
\usepackage{todonotes}
\usepackage[notion, quotation, silent]{knowledge}
%remove silent to have knowledge warnings
\usepackage{mathtools}
\usepackage{amssymb}
\usepackage{amsmath}
\usepackage{amsthm}
\usepackage{xspace}
\usepackage{tikz}
%\usepackage{array}
%\usepackage{changepage}
\usepackage{enumitem}
\usepackage[capitalise]{cleveref}
\usepackage{cite}



\usetikzlibrary{arrows,calc,automata,shapes,positioning}
\tikzset{AUT style/.style={>=angle 60,initial text= ,every edge/.append,every state/.style={minimum size=20,inner sep=2}}}

%%% Basic Math

\newcommand{\nats}{\mathbb{N}}
\newcommand{\set}[1]{\{#1\}}
\newcommand{\powerset}[1]{2^{#1}}
\newcommand{\step}[1]{\xrightarrow{#1}}
\newcommand{\size}[1]{|#1|}
\newcommand{\nset}[2]{\{#1,\dots,#2\}} %interval of natural numbers

%%% Complexity classes

\newcommand{\poly}{\text{\sc{P}}\xspace}
\newcommand{\conp}{\textsc{coNP}\xspace}
\newcommand{\pspace}{\text{\sc{PSpace}}\xspace}
\newcommand{\expt}{\textsc{ExpTime}\xspace}
\newcommand{\nexpt}{\textsc{NExpTime}\xspace}
\newcommand{\exps}{\textsc{ExpSpace}\xspace}


%%% Genral definition
\newcommand{\prot}{\mathcal{P}} %protocol
\newcommand{\form}{\mathbf{f}}
\newcommand{\config}{\gamma}
\newcommand{\run}{\rho}

\newcommand{\formapprox}[2]{\overline{#1}^{#2}}
\newcommand{\configapprox}[3]{\overline{#1}^{#2, #3}}

\newcommand{\FormApproxSet}[2]{\mathbf{FA}_{#1}^{#2}}


\begin{document}
	
	\maketitle
	
	
	\begin{abstract}
		We study IOPPUDs.
	\end{abstract}
	
	\section{Introduction}
	
	\section{Definitions}
	
	\begin{definition}
		A \emph{population protocol with unordered data} is a tuple $(Q, \delta, I, O)$ with $Q$ a finite set of states,
		$\Delta \subseteq Q^2 \times \set{=, \neq} \times Q^2$ a set of transitions,
		$I \subseteq Q$ a set of initial states, and
		$O : Q \to \set{\top, \bot}$ an output function.
		
		Here we are interested in \emph{immediate observation population protocols with unordered data} (IOPPUD), which is the subclass of protocols in which every transition is of the form $(q_1, q_2, \sim, q_1, q_3)$, with $q_1, q_2, q_3 \in Q$ and $\sim \in \set{=, \neq}$.
		We will denote such transitions by $q_2 \trans{q_1}{\sim} q_3$.
	\end{definition}
	
	We fix an infinite data domain $\Dataset$ and an infinite set of agents $\Agentset$. We could take them to be $\nats$, but this could cause some confusion with the other uses of $\nats$ in the definitions.
	
	A \emph{configuration} is a function $\config$ from $\Agentset$ to $Q \times \Dataset \cup \set{\bot}$.
	
	A \emph{step} $\config_1 \step{\sim}{a}{a_o} \config_2$ with $\config_1, \config_2 \in \configset$, $\sim \in \set{=, \neq}$ and $a, a_o \in \Agentset$ is defined when there exists  $\delta = q_1 \trans{q}{\sim} q_2 \in \Delta$ and $d, d_o \in \Dataset$ such that $\config_1(a) = (q_1, d)$, $\config_2(a) = (q_2, d)$ and $\config_1(a_o) = \config_2(a_o) = (q, d_o)$ and $\config_1(a') = \config_2(a')$ for all $a' \neq a$, and $d \sim d_o$.
	We say that agent $a$ \emph{observes} agent $a_o$. 
	


	\section{Bounds on the number of observed agents}
	
		\begin{definition}
		Let $\run : \config_1$ be a run.
		We say that agent is \emph{internally observed} in $\run$ if $\run$ contains a step of the form $\config_1 \step{=}{a}{a_o} \config_2$, and \emph{externally observed} if $\run$ contains a step of the form $\config_1 \step{\neq}{a}{a_o} \config_2$.
		We say that $a_o$ is \emph{observed} in $\run$ if it is either internally or externally observed.
	\end{definition}
	
	
	\begin{lemma}
		For all run $\run : \config_1 \xrightarrow{*} \config_2$ there exists a run $\run' : \config_1 \xrightarrow{*} \config_2$ such that for all $d \in \Dataset$, at most $\size{Q}^3$ agents with data $d$ are observed.
	\end{lemma}
	

	
	\begin{lemma}
		For all run $\run : \config_1 \xrightarrow{*} \config_2$ there exists another run $\run' : \config_1 \xrightarrow{*} \config_2$ such that for all $d \in \Dataset$, at most $\size{Q}^3$ agents with data $d$ are observed and at most $\size{Q}^{3\size{Q}^3}$\corto{Tentative bound} agents are externally observed.
	\end{lemma}
	 
	
	\section{Equivalence relation}
	
	Given a configuration $\config$ and a datum $d \in \Dataset$, we define the $n$-approximation of $d$ in $\config$ as the vector $\dataapprox{n}{\config}{d} : Q \to \nset{0}{n}$ such that for all $q\in Q$, $\dataapprox{n}{\config}{d}(q)=\min(\size{\set{a \in \Agentset \mid \config(a)= (q,d)}},n)$.
	
	We denote the set of all \textbf{non-zero} $n$-approximations by $\DataApproxSet{n}$
	Similarly, we define the $(n,M)$-approximation of a configuration $\config$ as the vector $\configapprox{n}{M}{\config} : \DataApproxSet{n} \to \nset{0}{M}$ such that for all $v\in \DataApproxSet{n}$,
	$\configapprox{n}{M}{\config}(v) = \min(\size{\set{d \in \nats \mid \dataapprox{n}{\config}{d} = v}},M)$.
	
	
	For all $n, M \in \nats$ we define the following equivalence relation on configurations: 
	For all $\config_1, \config_2$, we have $\config_1 \equiv_{n,M} \config_2$ if and only if $\configapprox{n}{M}{\config_1} = \configapprox{n}{M}{\config_2}$.
	
	

	\begin{lemma}
		For all $n, M \in \nats$, for all configurations $\config_1, \config_2, \config_1'$, if there is a run $\run : \config_1 \xrightarrow{*} \config_2$ and $\config_1 \equiv_{n\size{Q}+\size{Q}^3,Mn\size{Q}+\size{Q}^{3\size{Q}^3}} \config'_1$ then there exists $\config'_2$ such that $\config'_2 \equiv_{n,M} \config_2$ and there is a run $\run' : \config_1' \xrightarrow{*} \config_2'$.
	\end{lemma}
	
	\section{Logic or game interpretation (bounding quantifiers)}
	
	\section{Complexity}
	
	\section{Conclusion}
\end{document}
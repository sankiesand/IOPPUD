\documentclass[a4paper,UKenglish,cleveref, autoref, thm-restate]{lipics-v2021}
%This is a template for producing LIPIcs articles. 
%See lipics-v2021-authors-guidelines.pdf for further information.
%for A4 paper format use option "a4paper", for US-letter use option "letterpaper"
%for british hyphenation rules use option "UKenglish", for american hyphenation rules use option "USenglish"
%for section-numbered lemmas etc., use "numberwithinsect"
%for enabling cleveref support, use "cleveref"
%for enabling autoref support, use "autoref"
%for anonymousing the authors (e.g. for double-blind review), add "anonymous"
%for enabling thm-restate support, use "thm-restate"
%for enabling a two-column layout for the author/affilation part (only applicable for > 6 authors), use "authorcolumns"
%for producing a PDF according the PDF/A standard, add "pdfa"

%\pdfoutput=1 %uncomment to ensure pdflatex processing (mandatatory e.g. to submit to arXiv)
%\hideLIPIcs  %uncomment to remove references to LIPIcs series (logo, DOI, ...), e.g. when preparing a pre-final version to be uploaded to arXiv or another public repository

%\graphicspath{{./graphics/}}%helpful if your graphic files are in another directory

\bibliographystyle{plainurl}% the mandatory bibstyle

\title{Immediate Observation Population Protocols with Unordered Data}

\titlerunning{IOPPUD} %TODO optional, please use if title is longer than one line

%\author{Jane {Open Access}}{Dummy University Computing Laboratory, [optional: Address], Country \and My second affiliation, Country \and \url{http://www.myhomepage.edu} }{johnqpublic@dummyuni.org}{https://orcid.org/0000-0002-1825-0097}{(Optional) author-specific funding acknowledgements}%TODO mandatory, please use full name; only 1 author per \author macro; first two parameters are mandatory, other parameters can be empty. Please provide at least the name of the affiliation and the country. The full address is optional. Use additional curly braces to indicate the correct name splitting when the last name consists of multiple name parts.

\author{Joan R. Public}{Department of Informatics, Dummy College, Country}{joanrpublic@dummycollege.org}{}{}


\authorrunning{} %TODO mandatory. First: Use abbreviated first/middle names. Second (only in severe cases): Use first author plus 'et al.'

\Copyright{Joan R. Public} %TODO mandatory, please use full first names. LIPIcs license is "CC-BY";  http://creativecommons.org/licenses/by/3.0/

%\ccsdesc[100]{\textcolor{red}{Replace ccsdesc macro with valid one}}
%%TODO mandatory: Please choose ACM 2012 classifications from
%%https://dl.acm.org/ccs/ccs_flat.cfm

% \begin{CCSXML}
	% 	<ccs2012>
	% 	<concept>
	% 	<concept_id>10003752.10003753.10003761.10003763</concept_id>
	% 	<concept_desc>Theory of computation~Distributed computing models</concept_desc>
	% 	<concept_significance>500</concept_significance>
	% 	</concept>
	% 	</ccs2012>
	% \end{CCSXML}

\ccsdesc[500]{Theory of computation~Verification by model checking}
\ccsdesc[300]{Theory of computation~Distributed computing models}

% \ccsdesc{Theory of computation~Models of computation~Concurrency~Distributed computing models}

\keywords{Population protocols} %TODO mandatory; please add comma-separated list of keywords



%\category{} %optional, e.g. invited paper

%\relatedversion{} %optional, e.g. full version hosted on arXiv, HAL, or other respository/website
%\relatedversiondetails[linktext={opt. text shown instead of the URL}, cite=DBLP:books/mk/GrayR93]{Classification (e.g. Full Version, Extended Version, Previous Version}{URL to related version} %linktext and cite are optional

%\supplement{}%optional, e.g. related research data, source code, ... hosted on a repository like zenodo, figshare, GitHub, ...
%\supplementdetails[linktext={opt. text shown instead of the URL}, cite=DBLP:books/mk/GrayR93, subcategory={Description, Subcategory}, swhid={Software Heritage Identifier}]{General Classification (e.g. Software, Dataset, Model, ...)}{URL to related version} %linktext, cite, and subcategory are optional

%\funding{(Optional) general funding statement \dots}%optional, to capture a funding statement, which applies to all authors. Please enter author specific funding statements as fifth argument of the \author macro.

%\acknowledgements{I want to thank \dots}%optional

%\nolinenumbers %uncomment to disable line numbering



%Editor-only macros:: begin (do not touch as author)%%%%%%%%%%%%%%%%%%%%%%%%%%%%%%%%%%
% \EventEditors{John Q. Open and Joan R. Access}
% \EventNoEds{2}
% \EventLongTitle{42nd Conference on Very Important Topics (CVIT 2016)}
% \EventShortTitle{CVIT 2016}
% \EventAcronym{CVIT}
% \EventYear{2016}
% \EventDate{December 24--27, 2016}
% \EventLocation{Little Whinging, United Kingdom}
% \EventLogo{}
% \SeriesVolume{42}
\ArticleNo{1}
%%%%%%%%%%%%%%%%%%%%%%%%%%%%%%%%%%%%%%%%%%%%%%%%%%%%%%




\newcommand{\cortoin}[1]{\todo[color=blue!20,inline]{\small #1}}
\newcommand{\corto}[1]{\todo[color=blue!20]{\small #1}}

\newcommand{\nicoin}[1]{\todo[color=red!20,inline]{\small #1}}
\newcommand{\nico}[1]{\todo[color=red!20]{\small #1}}

\newcommand{\chanain}[1]{\todo[color=green!20,inline]{\small #1}}
\newcommand{\chana}[1]{\todo[color=green!20]{\small #1}}

\newcommand{\steffenin}[1]{\todo[color=purple!20,inline]{\small #1}}
\newcommand{\steffen}[1]{\todo[color=purple!20]{\small #1}}

\newcommand{\sandrain}[1]{\todo[color=orange!30,inline]{\small #1}}
\newcommand{\sandra}[1]{\todo[color=orange!30]{\small #1}}

\newif\ifproofs
\proofstrue

\newif\ifintuition
\intuitionfalse



\usepackage[utf8]{inputenc}
\usepackage{xcolor}
\usepackage{hyperref}
\usepackage{todonotes}
\usepackage[notion, quotation, silent]{knowledge}
%remove silent to have knowledge warnings
\usepackage{mathtools}
\usepackage{amssymb}
\usepackage{amsmath}
\usepackage{amsthm}
\usepackage{xspace}
\usepackage{tikz}
%\usepackage{array}
%\usepackage{changepage}
\usepackage{enumitem}
\usepackage[capitalise]{cleveref}
\usepackage{cite}



\usetikzlibrary{arrows,calc,automata,shapes,positioning}
\tikzset{AUT style/.style={>=angle 60,initial text= ,every edge/.append,every state/.style={minimum size=20,inner sep=2}}}

%%% Basic Math

\newcommand{\nats}{\mathbb{N}}
\newcommand{\set}[1]{\{#1\}}
\newcommand{\powerset}[1]{2^{#1}}
\newcommand{\step}[1]{\xrightarrow{#1}}
\newcommand{\size}[1]{|#1|}
\newcommand{\nset}[2]{\{#1,\dots,#2\}} %interval of natural numbers

%%% Complexity classes

\newcommand{\poly}{\text{\sc{P}}\xspace}
\newcommand{\conp}{\textsc{coNP}\xspace}
\newcommand{\pspace}{\text{\sc{PSpace}}\xspace}
\newcommand{\expt}{\textsc{ExpTime}\xspace}
\newcommand{\nexpt}{\textsc{NExpTime}\xspace}
\newcommand{\exps}{\textsc{ExpSpace}\xspace}


%%% Genral definition
\newcommand{\prot}{\mathcal{P}} %protocol
\newcommand{\form}{\mathbf{f}}
\newcommand{\config}{\gamma}
\newcommand{\run}{\rho}

\newcommand{\formapprox}[2]{\overline{#1}^{#2}}
\newcommand{\configapprox}[3]{\overline{#1}^{#2, #3}}

\newcommand{\FormApproxSet}[2]{\mathbf{FA}_{#1}^{#2}}
\definecolor{Blue Sapphire}{HTML}{002346} 
\definecolor{Gamboge}{HTML}{ee9b00}
\definecolor{Ruby Red}{HTML}{800000}

\IfKnowledgePaperModeTF{
	%
}{
	% If we are NOT in paper mode (i.e. in composition mode or electronic mode)
	\knowledgestyle{intro notion}{color={Ruby Red}, emphasize}
	\knowledgestyle{notion}{color={Blue Sapphire}}
	\hypersetup{
		colorlinks=true,
		breaklinks=true,
		linkcolor={Blue Sapphire}, % Links to sections, pages, etc.
		citecolor={Blue Sapphire}, % Links to bibliography
		filecolor={Blue Sapphire}, % Links to local file
		urlcolor={Blue Sapphire},
	}
}
\IfKnowledgeCompositionModeTF{
	% If we are in composition mode, highlight unknown stuff (in yellow) and display the anchor point.
	\knowledgeconfigure{anchor point color={Ruby Red}, anchor point shape=corner}
	\knowledgestyle{intro unknown}{color={Gamboge}, emphasize}
	\knowledgestyle{intro unknown cont}{color={Gamboge}, emphasize}
	\knowledgestyle{kl unknown}{color={Gamboge}}
	\knowledgestyle{kl unknown cont}{color={Gamboge}}
}{
	%
}


\knowledge{notion}
| population protocol with unordered data
| PPUD

\knowledge{notion}
| immediate observation PPUD
| IOPPUD

\knowledge{notion}
| configuration
| configurations

\knowledge{notion}
| step
| steps

\knowledge{notion}
| run
| runs

\knowledge{notion}
| observe
| observes

\knowledge{notion}
| observed
| observations
| non-observed

\knowledge{notion}
| internally observed
| internally

\knowledge{notion}
| externally observed
| externally

\knowledge{notion}
| shadows
| shadow

\knowledge{notion}
| specification
| specifications

\knowledge{notion}
| predicates
| predicate


\begin{document}
	
	\maketitle
	
	
	\begin{abstract}
		We study IOPPUDs.
	\end{abstract}
	
	\section{Introduction}
\label{sec:intro}

IOPPUD are an interesting restriction of an interesting extension of population protocols.
	
	\section{Definitions}
\label{sec:definitions}


\begin{definition}
	A ""population protocol with unordered data"" ("PPUD") is a tuple $(Q, \Delta, I, O)$ with $Q$ a finite set of states,
	$\Delta \subseteq Q^2 \times \set{=, \neq} \times Q^2$ a set of transitions,
	$I \subseteq Q$ a set of initial states, and
	$O : Q \to \set{\top, \bot}$ an output function.
	
	In this work we are interested in ""immediate observation PPUD"" ("IOPPUD"), which is the subclass of protocols in which every transition is of the form $(q_1, q_2, \sim, q_1, q_3)$, with $q_1, q_2, q_3 \in Q$ and $\sim \in \set{=, \neq}$.
	We will denote such transitions by $q_2 \trans{q_1}{\sim} q_3$.
\end{definition}

We fix an infinite data domain $\Dataset$, an infinite set of agents $\Agentset$ and a function $\intro*\dataof : \Agentset \to \Dataset$ such that $\dataof^{-1}(d)$ is infinite for all $d \in \Dataset$.



A ""configuration"" is a function $\config \colon \Agentset \to Q \cup \set{\bot}$.\steffen{such that $\config(a) \in Q$ for only finitely many $a \in \Agentset$? Otherwise, this could lead to problems below.} We write $\intro*\configset$ for the set of all "configurations". Given a configuration $\config$, we define $\counting{\config} \colon \Dataset \to \nats^Q$ such that for all $d \in \Dataset$ and $q \in Q$, it holds that $\counting{\config}(d)(q) = \size{\set{a \in \Agentset \mid \dataof(a) = d \land \config(a) =q}}$\steffen{Replaced $:$ by $\colon$. Also everyhwere below?}.


A ""step"" $\config_1 \step{\sim}{a}{a_o} \config_2$ with $\config_1, \config_2 \in \configset$, $\sim~ \in \set{=, \neq}$ and $a, a_o \in \Agentset$ is defined when there exists  $\delta = q_1 \trans{q}{\sim} q_2 \in \Delta$ and $d, d_o \in \Dataset$ such that $\config_1(a) = (q_1, d)$\steffen{$\config_1(a) = q_1$, $\dataof(a) = d$? Also everywhere else in this paragraph.}, $\config_2(a) = (q_2, d)$ and $\config_1(a_o) = \config_2(a_o) = (q, d_o)$ and $\config_1(a') = \config_2(a')$ for all $a' \neq a$, and $d \sim d_o$.
\steffenin{Alternative: For every $\config_1, \config_2 \in \configset$, $\sim~ \in \set{=, \neq}$, and $a, a_o \in \Agentset$, we write $\config_1 \step{\sim}{a}{a_o} \config_2$, called a ""step"", to say that there exists $\delta = q_1 \trans{q}{\sim} q_2 \in \Delta$ such that $\config_1(a) = q_1$, $\config_2(a) = q_2$, $\config_1(a_o) = \config_2(a_o) = q$, $\config_1(a') = \config_2(a')$ for all $a' \in \Agentset \setminus \{a\}$, and $\dataof(a) \sim \dataof(a_o)$.}
We say that agent $a$ ""observes"" agent $a_o$.
We will write $\config_1 \to \config_2$ to say that there exist $a, a_0$ and $\sim$ such that $\config_1 \step{\sim}{a}{a_o} \config_2$.

A ""run"" $\run$ is a sequence of consecutive "steps" $\run: \config_0 \to \config_1 \to \cdots \to  \config_m$.\steffen{$\run:$ or $\run =$? Consistent with prefix runs?}
Further, for all $i \in \nset{1}{m}$, we define the prefix run $\prefixrun{\run}{i} = \config_0 \to \config_1 \to \cdots \to  \config_i$ and suffix run $\suffixrun{\run}{i} = \config_i \to \config_{i+1} \to \cdots \to  \config_m$. We write $\run : \config \xrightarrow{*} \config'$ to say that $\run$ goes from $\config$ to $\config'$.\steffen{We write $\run : \config \xrightarrow{*} \config'$ to say that there are $m \in \nats$ and $\config_0, \dots, \config_m \in \configset$ such that $\config = \config_0$, $\config' = \config_m$, and $\run: \config_0 \to \config_1 \to \cdots \to \config_m$? (or $\run = \dots$?)}

For all "configurations" $\config$, we let $\Reach(\config) \coloneqq \set{\config' \in \configset \mid \exists \run : \config \xrightarrow{*} \config'}$.


\begin{definition}
	Let $\run : \config_1 \xrightarrow{*} \config_2$ be a "run", $d\in \Dataset$, and let $\Agentset^d_o$ be the agents with datum $d$ that are "observed" in $\run$. For all $q_1, q_2 \in Q$ let $\Agentset^d_{q_1, q_2}$ be the set of agents with datum $d$ that start in $q_1$ and end in $q_2$, that is, $\Agentset^d_{q_1, q_2} = \bigl\{a \in \Agentset \ \mid \ \dataof(a) = d, \config_1(a) = q_1, \config_2(a) = q_2\bigr\}$.
	\steffen{Below, we use the definitions for different runs. Include $\run$ in the notation somewhere? Maybe $\Agentset^{\rho, d}_{q_1, q_2}$}
	
	We define the ""trace"" of $d$ in $\run$ as the function $\trace{d}{\run} : Q^2 \to \nats$ such that for all $q_1, q_2 \in Q$,	$\trace{d}{\run}(q_1, q_2) = \size{\Agentset^d_{q_1, q_2}}$
		
	
	\AP We define the ""shadow"" of $d$ in $\run$ as the function $\shadow{d}{\run} : Q^2 \to \nats \cup \set{\bot}$ such that for all $q_1, q_2 \in Q$,
	\begin{equation}
		\shadow{d}{\run}(q_1, q_2) =
		\left\{
		\begin{aligned}
			&\bot &\text{ if } \Agentset^d_{q_1,q_2} =\emptyset\\
			&\size{\Agentset^d_o \cap \Agentset^d_{q_1, q_2}} &\text{ otherwise.}
		\end{aligned}
		\right.
	\end{equation}
\end{definition}
%	
The "trace" and "shadow" of $d$ describe the flow of its agents between states in the "run". For each pair of states $q_1, q_2$, the "trace" counts the number of agents going from $q_1$ to $q_2$ while
the "shadow" simply indicates if some agents went from $q_1$ to $q_2$ and counts the "observed" ones among them. The idea behind the "shadow" is that if two data $d, d'$ have the same "shadow" in $\run$ and no agent of $d'$ is "externally observed"\steffen{only defined below}, then we can make each agent of $d'$ copycat\steffen{not defined yet} an agent of $d$ so that in the end, they all reach the same end configuration\steffen{“Reaching the same configuration” sounds a bit fishy.}.


\begin{definition}
	A ""predicate"" is a Boolean combination of formulas of the form $\exists datum_1, \ldots, datum_k, \psi$, where $\psi$ is a Boolean combination of inequalities of the form $\#(q,datum_i) \leq B$ with $q\in Q$, $i \in \nset{1}{k}$, and $B \in \nats$.
	\steffen{shorten $datum_i$? (I don't have a good idea, yet)}
	
	A ""specification"" is a formula of the form $Q_1 \config_1  \in \Reach(\config_0), \ldots, Q_p \config_p \in \Reach(\config_{p-1}), \phi$, where $\config_0, \ldots, \config_{p}$ are variables, $Q_i$ is a quantifier\steffen{$Q_i \in \{\exists,\forall\}$?} for all $i \in \nset{1}{p}$, and $\phi$ is a Boolean combination of pairs $(\config_i, \psi)$ with $i \in \nset{1}{p}$ and $\psi$ a "predicate".
	
	\cortoin{Shape to discuss}
\end{definition}

A "predicate" $\exists datum_1, \ldots, datum_k, \psi$ is ""satisfied"" by a configuration $\config$ if there exist \textbf{distinct} data $d_1, \ldots, d_n \in \Dataset$ such that $\psi$ is satisfied by replacing each $\#(q,datum_i) \leq B$ by $\top$ if $\counting{\config}(d_i)(q) \leq B$ and $\bot$ otherwise.
\steffen{Move into the definition?}


	%!TEX root = main.tex
\section{Bounds on the number of observed agents}
\label{sec:bounds-observed-agents}

\begin{definition}
	Let $\run : \config_1 \xrightarrow{*} \config_2$ be a "run".
	We say that agent $a_0$ is ""internally observed"" in $\run$ if $\run$ contains a step of the form $\config_1 \step{=}{a}{a_o} \config_2$, and ""externally observed"" if $\run$ contains a step of the form $\config_1 \step{\neq}{a}{a_o} \config_2$.
	We say that $a_o$ is ""observed"" in $\run$ if it is either "internally" or "externally observed", and that a datum $d$ is observed if an agent $a$ with $\dataof(a) =d$ is. 
\end{definition}

\subsection{Bounds on the number of observed agents per datum}

We call  ""trajectory"" of an agent $a$ in a run $\run=C_1 \xrightarrow{} C_2 \xrightarrow{} \ldots \xrightarrow{} C_n$ 
the sequence $q_1 \ldots q_n$ of states of $Q$ such that  $C_i(a)=q_i$ for all $i$.

\begin{lemma}[Agents core lemma]
	\label{lem:agents-core-lemma}
	Let $\run : \config_1 \xrightarrow{*} \config_2$ be a "run" with $\Agentset_{\run}$ the set of agents appearing in it. There exists a "run" $\run' : \config'_1 \xrightarrow{*} \config'_2$ over a subset of agents $\Agentset_{\run'} \subseteq \Agentset_{\run}$ such that:
	\begin{itemize}
		\item for all $a \in \Agentset_{\run'}$, $\config_1(a) = \config'_1(a)$ and $\config_2(a) = \config'_2(a)$,
		
		\item and for all $d \in \Dataset$, $q_1, q_2 \in Q$, if there exists an agent in $\Agentset_{\run}$ such that $\config_1(a) = (q_1, d)$ and $\config_2(a) = (q_2, d)$ then there is such an agent in $\Agentset_{\run'}$.
		
		\item For all $d \in \Dataset$, there are at most $\size{Q}^3$ agents with datum $d$ in $\Agentset_{\run'}$.
		%\cortoin{we can achieve $\size{Q}^3$ but I think it makes the proof a little more technical}
	\end{itemize}
\end{lemma}

\begin{proof}
%	We mimic the \emph{bunch} argument from \cite{EsparzaRW2019}. 
%	Let $d\in \Dataset$ a datum appearing in $\run$, and let $q_1, q_2 \in Q$, and $\Agentset^d_{q_1, q_2} = \set{a \in \Agentset_{\run} \mid \dataof(a) = d, \config_1(a) = q_1, \config_2(a) = q_2}$. Suppose $\size{\Agentset^d_{q_1, q_2}} \geq 3\size{Q}$, then let $S$ be the set of states visited by agents of $\Agentset^d_{q_1, q_2}$ in $\run$.
%	
%	For all $q \in S$ let $\alpha_q$ be the first agent of $\Agentset^d_{q_1, q_2}$ to reach $q$ (for $q_1$ we pick one arbitrarily), and $\beta_q$ the last one to leave it (we pick an arbitrary agent for $\beta_{q_2}$). 
%	Note that those agents do not have to be distinct.
%	
%	We pick a family of distinct agents $(a_q)_{q \in S}$ in $\Agentset^d_{q_1, q_2}$ that is disjoint from $(\alpha_q)_{q \in S}$ and $(\beta_q)_{q \in S}$ (we can do so as $\size{\Agentset^d_{q_1, q_2}} \geq 3\size{Q}$).
%	We define a new "run" $\overline{\run}$, where all agents not in $\Agentset^d_{q_1, q_2}$, as well as all $\alpha_q$ and $\beta_q$ behave the same. The agents of $\Agentset^d_{q_1, q_2}$ besides the $\alpha_q, \beta_q$ and $a_q$ are deleted.
%	Each $a_q$ mimics the transitions taken by $\alpha_q$ until it reaches $q$, then stays idle until $\beta_q$ reaches $q$ for the last time, after which it mimics $\beta_q$. \cortoin{informal, formal proof by induction on the "run"}
%	
%	Whenever an agent "observes" an agent of $\Agentset_{q_1, q_2}$ in state $q$, as this must happen in $\run$ between the arrival of $\alpha_q$ and the departure of $\beta_q$, $a_q$ is in $q$ at this point in $\overline{\run}$ and can hence be "observed" instead. Therefore all "observations" are still possible, and only agents of $(a_{q})_{q\in S}$ are "observed" among $\Agentset_{q_1, q_2}$.
%	
%	The start and end "configurations" are the same as before for all remaining agents as all $a_q$ still go from $q_1$ to $q_2$, and other agents behave the same.
%	
%	By applying this transformation on all data appearing in $\run$ and all pairs of state $q_1, q_2$, we obtain a "run" $\run'$ in which for all $d$, $q_1$, $q_2$, at most $3\size{Q}$ agents going from $q_1$ to $q_2$ with datum $d$ remain (either there are less than $3\size{Q}$ such agents, or we ensured than at most $\size{Q}$ of them are "observed").
%	In total, for all $d$, at most $3\size{Q}^3$ agents per datum remain.
%%%%%%%%%%%%%%%%%%%%%%%%%%%%%%   old proof above
We mimic the \emph{bunch} argument from \cite{EsparzaRW2019}. \chana{old proof commented out}
Let $\run$ be the run $\config_1=C_1 \xrightarrow{} C_2 \xrightarrow{} \ldots \xrightarrow{} C_n = \config_2$.
Let $d\in \Dataset$ a datum appearing in $\run$,  let $q_1, q_2 \in Q$, 
Suppose $\size{\Agentset^d_{q_1, q_2}} \geq \size{Q}$, and 
let $\bunchreach$ be the set of states visited by agents of $\Agentset^d_{q_1, q_2}$.
We are going to define a family $(a_q)_{q \in \bunchreach}$ of agents in $\Agentset^d_{q_1, q_2}$ 
such that the run with only  $(a_q)_{q \in \bunchreach}$ in $\Agentset^d_{q_1, q_2}$ 
and equal to $\run$ everywhere else is still a valid run,
Repeating the operation for every ($\size{Q}^2$) set of agents $\Agentset^d_{q_1, q_2}$
will yield a run fulfilling the Lemma conditions.

For all $q \in \bunchreach$, 
let $f$ be the first moment $q$ is reached in $\run$, 
i.e. the minimal index such that there exists an $a\in \Agentset^d_{q_1, q_2}$
with $C_f(a)=q$.
Let $l$ be the last moment $q$ is reached in $\run$, 
i.e. the maximal index such that there exists an $a\in \Agentset^d_{q_1, q_2}$
with $C_l(a)=q$.
Let $\alpha_q$ be one of the agent in $\Agentset^d_{q_1, q_2}$ that reaches $q$ first,
i.e. $C_f(\alpha_q)=q$, and 
let $\beta_q$ be one of the agent in $\Agentset^d_{q_1, q_2}$ that reaches $q$ last
i.e. $C_l(\beta_q)=q$. 
Note that these agents do not have to be distinct.
We pick an agent $a_q \in \Agentset^d_{q_1, q_2}$ and modify its trajectory as follows.
Agent $a_q$ mimics the transitions taken by $\alpha_q$ until $f$, 
then stays idle until $l$, after which it mimics $\beta_q$. 
We can choose the $a_q$ distinct for each $q\in \bunchreach$, since $\size{\Agentset^d_{q_1, q_2}} \geq \size{Q}$.

We modify $\run$ by deleting all agents of $\Agentset^d_{q_1, q_2}$ that are not in $(a_q)_{q \in \bunchreach}$.
This is still a valid run, because whenever an agent "observes" an agent of $\Agentset^d_{q_1, q_2}$ in state $q$, 
 this must happen at moment $i$ between $f$ and $l$. 
 Our definition of $a_q$'s trajectory ensures it  is in $q$ at $i$ and can hence be "observed" instead. 
 Therefore all "observations" are still possible.
	
By applying this transformation on all data appearing in $\run$ and all pairs of state $q_1, q_2$, 
we obtain a "run" $\run'$ in which for all $d$, $q_1$, $q_2$, 
at most $\size{Q}$ agents going from $q_1$ to $q_2$ with datum $d$ remain 
In total, for all $d$, at most $\size{Q}^3$ agents per datum remain.
\chana{tbd: show $\run'$ fulfills the lemma conditions}
\chana{other option: for all $d$, treat $\run_d$ (only the $d$-agents of $\run$) as a classic IO run, and apply the results of \cite{EsparzaRW2019}}
\end{proof}



\begin{lemma}[Agents copycat lemma]
	\label{lem:agents-copycat}
	Let  $\run : \config_1 \xrightarrow{*} \config_2$ a "run", let $q_1, q_2 \in Q$,  $d \in \Dataset$, and  $\Tilde{a} \in \Agentset$ such that $\dataof(\Tilde{a}) = d$ and $\config_1(\Tilde{a}) = \bot$. 
	
	If there exists an agent $a$ such that $\dataof(a)=d$, $\config_1(a) = q_1$ and $\config_2(a) = q_2$, then there exists a run $\Tilde{\run} : \Tilde{\config}_1 \xrightarrow{*} \Tilde{\config}_2$ such that
	\begin{itemize}
		\item $\Tilde{\config}_1(\Tilde{a}) = q_1$, and $\Tilde{\config}_1(a') = \config_1(a')$ for all $a' \neq \Tilde{a}$.
		
		\item $\Tilde{\config}_2(\Tilde{a}) = q_2$, and $\Tilde{\config}_2(a') = \config_2(a')$ for all $a' \neq \Tilde{a}$.
		
		\item $\Tilde{a}$ is not observed in $\Tilde{\run}$.
	\end{itemize}
\end{lemma}

\begin{proof}
	We set $\Tilde{\config}_1$ such that $\Tilde{\config}_1(\Tilde{a}) = q_1$ and $\Tilde{\config}_1(a') = \config_1(a')$ for all $a' \neq \Tilde{a}$.
	
	We then construct $\Tilde{\run}$ by inserting after every step $\step{\sim}{a}{a_o}$ in $\run$ during which $\Tilde{a}$ moves a step
	$\step{\sim}{\Tilde{a}}{a_o}$.
	We can do so because $\dataof(\Tilde{a})=\dataof(a)$, and by immediate induction they are in the same state before every step $\step{\sim}{\Tilde{a}}{a_o}$.
	
	The configuration $\Tilde{\config}_2$ reached by $\Tilde{\run}$ is such that $\Tilde{\config}_2(\Tilde{a}) = \Tilde{\config}_2(a) = q_2$ and $\Tilde{\config}_1(a') = \config_1(a')$ for all $a' \neq \Tilde{a}$. The two first items of the lemma are trivially satisfied.
	For the third one, observe that $\Tilde{a}$ did not appear in $\run$ and that we did not add any transition where $\Tilde{a}$ is observed.
\end{proof}


\begin{corollary}
	For all "run" $\run : \config_1 \xrightarrow{*} \config_2$, there exists a "run" $\Tilde{\run} : \config_1 \xrightarrow{*} \config_2$ such that for all $d \in \Dataset$,at most $3\size{Q}^3$ agents with datum $d$ are "observed" in $\run$.
\end{corollary}

\begin{proof}
	Take the "run" $\run'$ from~\cref{lem:agents-core-lemma} and add the missing agents using \cref{lem:agents-copycat} by making them copycat an agent with the same datum, initial and final state from $\Agentset_{\run'}$. Such an agent exists by the second item of the lemma.
\end{proof}

\subsection{Bounds on the number of observed data}

\begin{lemma}[Data core lemma]
	\label{lem:data-core-lemma}
	Let $\run : \config_0 \to \config_2 \to \cdots \to \config_m$ be a "run"  over a set of data $\Dataset_{\run}$, and let $K$ be such that there are at most $K$ "observed" agents of each datum "observed" in the "run". There exists another "run" $\run' : \config_0 \xrightarrow{*} \config_m$ and a subset of data $\Dataset_{\run'}$ such that:
	\begin{itemize}
		\item for all agents $a$ of datum $d \in \Dataset_{\run'}$, $\config_1(a) = \config'_1(a)$ and $\config_2(a) = \config'_2(a)$,
		
		\item for all $d \in \Dataset_{\run}$, there exists $d' \in \Dataset_{\run'}$ such that $\shadow{d'}{\run'} = \shadow{d}{\run}$.
		
		\item $\size{\Dataset_{\run'}} \leq 4(K+2)^{2\size{Q}^2}$.
	\end{itemize}
\end{lemma}


\begin{proof}
	We proceed similarly to the proof of \cref{lem:agents-core-lemma}. 
	Let $\Dataset_\run$ be the set of data appearing in $\run$.
	For all $d$ let $\Agentset_d$ be its set of agents and $\Agentset_{d,o}$ the subset of those agents that are "observed" at some point in the "run".
	
	As no more than $K$ agents of each datum are "observed" through $\run$, the "shadows" of all data are bounded by $K$ and thus there are at most $(K+2)^{\size{Q}^2}$ different ones. This also holds for all prefixes and suffixes of $\run$.
	
	Let $M=(K+2)^{2\size{Q}^2}$. 
	For all $d \in \Dataset_\run$ and $i \in \nset{1}{m}$ we define $\sigma(d,i) = (\shadow{d}{\prefixrun{\run}{i}}, \shadow{d}{\suffixrun{\run}{i}})$. This functions takes a datum and a point in the run and returns the "shadow" of the past and the future of that datum through the run. 
	
	We now lift the proof of \cref{lem:agents-core-lemma} from agents to data.
	Let $\ashadow : Q^2 \to \nset{0}{K}\cup \bot$ and let $\Dataset_{\ashadow} = \set{d \in \Dataset_{\run} \mid \shadow{d}{\run} = \ashadow}$. 
	Suppose $\size{\Dataset_{\ashadow}} \geq 4M$. 
	Let $S = \set{\sigma(d,i) \mid d \in \Dataset_{\ashadow}, i\in\nset{1}{m}}$. Note that $\size{S}\leq M$.
	
	For each $s \in S$ define $\alpha_s$ as the datum reaching $s$ first (i.e., such that $\sigma(\alpha_s,i) = s$ for a $i$ as small as possible) in $\run$ and $\beta_s$ as the last one.
	Then as $\size{\Dataset_{\ashadow}}\geq 4M$, we can pick for each $s \in S$ some data $d_s, e_s \in \Dataset_{\ashadow}$ such that the $(d_s)_{s\in S}, (e_s)_{s\in S}$ are disjoint and disjoint from the $(\alpha_s)_{s\in S}$ and $(\beta_s)_{s\in S}$.
	
	We create a new "run" $\overline{\run}$ where all agents with datum in $\set{\alpha_s, \beta_s \mid s \in S} \cup (\Dataset \setminus \Dataset_{\ashadow})$ behave the same. 
	The agents with datum in $\Dataset_{\ashadow} \setminus \set{\alpha_s, \beta_s, d_s, e_s \mid s \in S}$ are deleted.
	
	Let $s = (\ashadow_1, \ashadow_2) \in S$, we make agents of $d_s$ and $e_s$ follow the ones of $\alpha_s$ until it reaches $s$, then stay idle until $\beta_s$ last reaches $s$, and then follow $\beta_s$.
	This is possible as they have the same "shadows", hence we can match "observed" agents of $d_s$ (and $e_s$) to the ones of $\alpha_s$ (resp. $\beta_s$) one-to-one while preserving the states of departure and arrival, and the "non-observed" of $d_s$ and $e_s$ can follow agents of $\alpha_s$ (resp. $\beta_s$) that have the same starting and finishing states.
	
	When an agent with datum $d$ is "externally observed" at step $i$ in $\run$, we consider $s = \sigma(d,i)$ and make the moving agent "observe" $d_s$ or $e_s$ instead (one of them has to be different from $d$).
	
	We apply this transformation on all "shadows".
\end{proof}




\begin{lemma}[Data copycat lemma]
	\label{lem:data-copycat}
	Let  $\run : \config_1 \xrightarrow{*} \config_2$ a "run", let $\ashadow : Q^2 \to \nats \cup \set{\bot}$ a "shadow", $\atrace$ a trace such that $\atrace(q_1, q_2) \geq \ashadow(q_1, q_2)$ for all $q_1, q_2$ such that $\ashadow(q_1, q_2) \in \nats$ and let $\Tilde{d} \in \Dataset$ such that $\config_1$ does not contain any agent with datum $\Tilde{d}$. 
	
	If there exists a datum $d$ such that $\shadow{\run}{d}=\ashadow$ then there exists a run $\Tilde{\run} : \Tilde{\config}_1 \xrightarrow{*} \Tilde{\config}_2$ such that
	\begin{itemize}
		\item $\shadow{\Tilde{\run}}{\Tilde{d}} = \ashadow$ and $\shadow{\Tilde{\run}}{d'} = \shadow{\Tilde{\run}}{d'}$ for all $d' \neq \Tilde{d}$.
		
		\item $\trace{\run}{\Tilde{d}} = \atrace$ and $\trace{\run}{d'} = \trace{\Tilde{\run}}{d'}$ for all $d' \neq \Tilde{d}$.
	\end{itemize}
\end{lemma}

\begin{corollary}
	\label{cor:run-few-observed-data-and-agents}
	For all "run" $\run : \config_1 \xrightarrow{*} \config_2$, there exists a "run" $\Tilde{\run} : \config_1 \xrightarrow{*} \config_2$ such that for all $d \in \Dataset$,at most $3\size{Q}^3$ agents with datum $d$ are "observed" in $\run$, and agents of at most $4(3\size{Q}^3+2)^{2\size{Q}^2}$ different data are "observed".
\end{corollary}



	
	\section{Equivalence relation}
\label{sec:equivalence-relation}

Given a configuration $\config$ and a datum $d \in \Dataset$, we define the $n$-approximation of $d$ in $\config$ as the vector $\dataapprox{n}{\config}{d} : Q \to \nset{0}{n}$ such that for all $q\in Q$, $\dataapprox{n}{\config}{d}(q)=\min(\counting{\config}(d)(q),n)$.

For all $n \in \nats$ we define the following equivalence relations on $\configset \times \Dataset$: 

For all $(\config_1, d_1), (\config_2, d_2) \in \configset\times \Dataset$, we have $(\config_1, d_1) \intro*\dataequiv{n} (\config_2, d_2)$ if and only if $\dataapprox{n}{\config_1}{d_1} = \dataapprox{n}{\config_2}{d_2}$.



We denote the set of all \textbf{non-zero} $n$-approximations by $\DataApproxSet{n}$
Similarly, we define the $(n,M)$-approximation of a configuration $\config$ as the vector $\configapprox{n}{M}{\config} : \DataApproxSet{n} \to \nset{0}{M}$ such that for all $v\in \DataApproxSet{n}$,
$\configapprox{n}{M}{\config}(v) = \min(\size{\set{d \in \nats \mid \dataapprox{n}{\config}{d} = v}},M)$.

For all $n, M \in \nats$ we define the following equivalence relations on configurations: 

For all $\config_1, \config_2$, we have $\config_1 \intro*\configequiv{n}{M} \config_2$ if and only if $\configapprox{n}{M}{\config_1} = \configapprox{n}{M}{\config_2}$.

\begin{lemma}
	Let $n,M \in \nats$ and $\config, \chi$ be configurations such that $\config \equiv_{n,M} \chi$, and let $\psi$ a predicate with at most $M$ variables and such that all bounds appearing in inequalities are at most $n$.
	Then $\config$ satisfies $\psi$ if and only if $\chi$ does.  
\end{lemma}


For all $n, M \in \nats$ let $f(n) = (n+3\size{Q}^3)\size{Q}$ and $g(n,M) = (M+4(3\size{Q}^3+2)^{\size{Q}^2})(n+1)^{\size{Q}}$.

\begin{lemma}
	For all $n, M \in \nats$, for all configurations $\config_1, \config_2, \chi_1$, if there is a run $\run : \config_1 \xrightarrow{*} \config_2$ and $\config_1 \equiv_{f(n),g(n,M)} \chi_1$ then there exist $\chi_2$ such that $\chi_2 \equiv_{n,M} \config_2$ and a run $\pi : \chi_1 \xrightarrow{*} \chi_2$.
\end{lemma}

\begin{proof}
	Let us fix $K=3\size{Q}^3$. By \cref{cor:run-few-observed-data-and-agents}, there exists a run $\run' : \config_1 \xrightarrow{*} \config_2$, with $\leq K$ agents per datum and $\leq 4(K+2)^{2\size{Q}^2}$ data.
	
	We are going to define maps $\mu : \Dataset(\chi_1) \to \Dataset(\config_1)$ and $\nu : \Agentset(\chi_1) \to \Agentset(\config_1)$.
	
	We start with $\mu$: For all $\equiv_{f(n)}$-equivalence class $E$ and $\equiv_{n}$-equivalence class $F$, we write $\Dataset_{E,F}$ for the set of data $d \in \Dataset(\config_1)$ such that $(\config_1, d) \in E$ and $(\config_2, d) \in F$.
	
	We then define a subset $\overline{\Dataset_{E,F}}$ of data.
	If $\size{\Dataset_{E,F}} \leq M$ we set $\Dataset_{E,F} = \overline{\Dataset_{E,F}}$. Otherwise, we select a subset of $M$ arbitrary elements, to which we add all data in $\Dataset_{E,F}$ that are "observed" in $\run'$ to get $\overline{\Dataset_{E,F}}$.
	In the end we get a subset $\overline{\Dataset_{E,F}}$ of $\Dataset_{E,F}$ of size at most $M+4(K+2)^{2\size{Q}^2}$.
	
	For all $\equiv_{f(n)}$-equivalence class $E$, we define $\overline{\Dataset_{E}} = \bigcup_{F\text{ an }\equiv_n-\text{equivalence class}} \overline{\Dataset_{E,F}}$, which is a set of size at most $(M+4(K+2)^{2\size{Q}^2})(n+1)^{\size{Q}} = g(n,M)$, as there are $(n+1)^{\size{Q}}$ $\equiv_n$-equivalence classes.
	
 Since $\config_1 \equiv_{f(n),g(n,M)} \chi_1$, there must be at least $\size{\overline{\Dataset_{E}}}$ data in $E \cap \Dataset(\chi_1)$. We can thus map each datum in $E \cap \Dataset(\chi_1)$ to one in $\overline{\Dataset_{E}}$ so that all $d \in \overline{\Dataset_{E}}$ have an antecedent.
 
 We define $\mu$ as this mapping on all equivalence classes $E$.
	
For $\nu$, let $d \in \Dataset(\chi_1)$. We have, by definition of $\mu$, $\chi_1, d \equiv_{f(n)} \config_1, \mu(d)$. Like before, we define $\Agentset_{ q_1, q_2}$ as the set of agents with datum $\mu(d)$ that are in $q_1$ in $\config_1$ and $q_2$ in $\config_2$.
We then define a subset $\overline{\Agentset_{q_1, q_2}}$ of those agents. If $\size{\Agentset_{q_1, q_2}} \geq n$, then $\overline{\Agentset_{q_1, q_2}} = \Agentset_{q_1, q_2}$. Otherwise, we select a set of $n$ agents in $\Agentset_{q_1, q_2}$ and add all agents of $\Agentset_{q_1, q_2}$ "observed" in $\run'$ to it. In both cases $\size{\overline{\Agentset_{q_1, q_2}}} \leq n+3\size{Q}^3$.
We then set $\overline{\Agentset_{q_1}} = \bigcup_{q_2 \in Q} \overline{\Agentset_{q_1, q_2}}$. Clearly $\size{\overline{\Agentset_{q_1}}} \leq \size{Q}(n+3\size{Q}^3)=f(n)$. 

\cortoin{To be continued}
\end{proof}
	
	\section{Logic or game interpretation (bounding quantifiers)}
\label{sec:quantifier-bounds}


\begin{definition}
	A ""predicate"" is a boolean combination of formulas of the form $\exists d_1, \ldots, d_k, \psi$ with $\psi$ a boolean combination of inequalities of the form $\#(q,d_i) \leq B$ with $q\in Q$, $i \in \nset{1}{k}$ and $B \in \nats$.
	
	A ""specification"" is a formula of the form $Q_1 \config_1, Q_2 \config_2  \in \Reach(\config_1), \ldots, Q_p \config_p \in \Reach(\config_{p-1}), \phi$, where for each $i \in \nset{1}{p}$, $Q_i$ is a quantifier and $\phi$ is a boolean combination of pairs $(\config_i, \psi)$ with $i \in \nset{1}{p}$ and $\psi$ a "predicate".
\end{definition}

\begin{definition}
	Let $Q_1 \config_1, Q_2 \config_2  \in \Reach(\config_1), \ldots, Q_p \config_p \in \Reach(\config_{p-1}), \phi$ be a "specification". We define the following game between two players (called Eve and Adam.\cortoin{Spoiler/Duplicator?})
	
	The game goes as follows: for each $i \in \nset{1}{p}$, iteratively, a player (Eve if $Q_i = \exists$, Adam otherwise) chooses a "configuration" $\config_i$. 
	If $i>1$, that "configuration" must be in $\Reach(\config_{i-1})$. If it is not, the player loses.
	
	After all $\config_1, \ldots, \config_p$ have been chosen, we check if the formula $\phi$ is satisfied by them. If it is, Eve wins, otherwise Adam does.
\end{definition}

\begin{lemma}
	A "specification" is satisfied if and only if Eve wins the associated game.
\end{lemma}

\begin{lemma}
	For all $p \in \nats$ there exists $n_p, M_p \in \nats$, at most exponential in $\size{Q}$ and $p$, \cortoin{bound to specify} such that Eve wins the game defined above if and only if she wins the version where both players are restricted to "configurations" with at most $Mn$ agents.
\end{lemma}


	
	\section{Complexity}
\label{sec:complexity}

\begin{proposition}
	The following problem is decidable in \pspace:
	Given a "PPUD" $\mathcal{P}$ and two "configurations" $\config_1, \config_2$, is there a "run" from $\config_1$ to $\config_2$?
\end{proposition}




\begin{theorem}
	The following problem is decidable in \exps for all fixed $k$:
	
	Given an "IOPPUD" $\mathcal{P}$ and a "specification" $\phi$ with $k$ quantifiers, does $\mathcal{P}$ satisfy $\phi$?
\end{theorem}

\begin{theorem}
	The well-specification problem is \nexpt-hard for "IOPPUD".
\end{theorem}

\begin{corollary}
	Correctness and well-specification are decidable in \exps and \nexpt-hard for "IOPPUD".
\end{corollary}

\begin{theorem}
	The well-specification problem is undecidable for "PPUD".
\end{theorem}

	
	\section{Conclusion}
	\label{sec:conclusion}
	
	\bibliography{biblio}
\end{document}
